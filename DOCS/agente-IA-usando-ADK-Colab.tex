\documentclass[11pt]{article}
\usepackage[utf8]{inputenc}
\usepackage[spanish]{babel}
\usepackage{geometry}
\geometry{a4paper, margin=2cm}
\usepackage{listings}
\usepackage{xcolor}
\usepackage{hyperref}
\usepackage{graphicx}

\definecolor{codebackground}{rgb}{0.95,0.95,0.92}
\definecolor{codecomment}{rgb}{0.5,0.5,0.5}
\definecolor{codekeyword}{rgb}{0,0,0.5}
\definecolor{codestring}{rgb}{0.58,0,0.82}

\lstdefinestyle{colabstyle}{
    backgroundcolor=\color{codebackground},
    commentstyle=\color{codecomment},
    keywordstyle=\color{codekeyword},
    numberstyle=\tiny\color{codecomment},
    stringstyle=\color{codestring},
    basicstyle=\ttfamily\footnotesize,
    breakatwhitespace=false,
    breaklines=true,
    captionpos=b,
    keepspaces=true,
    numbers=left,
    numbersep=5pt,
    showspaces=false,
    showstringspaces=false,
    showtabs=false,
    tabsize=2,
    frame=single,
    framerule=0.5pt
}

\title{Guía para Desarrollo de Agentes de IA en Google Colab con Google ADK}
\author{}
\date{\today}

\begin{document}

\maketitle

\section{Requerimientos Previos}
\subsection{Requerimientos de Hardware en Google Colab}
\begin{itemize}
    \item \textbf{Tipo de entorno}: Se recomienda utilizar el entorno de ejecución con GPU (NVIDIA Tesla T4, P100, V100 o similar) para tareas que involucren modelos grandes.
    \item \textbf{Memoria RAM}: Mínimo 12 GB (preferiblemente 25+ GB para agentes complejos).
    \item \textbf{Almacenamiento}: El espacio gratuito es limitado ($\sim$80~GB). Para proyectos grandes, considere montar Google Drive.
\end{itemize}

\subsection{Requerimientos de Software y Configuración}
\begin{itemize}
    \item \textbf{Cuenta de Google}: Necesaria para acceder a Colab y Google Cloud APIs.
    \item \textbf{Google Cloud Project}: Se requiere un proyecto en Google Cloud Platform (GCP) con las APIs habilitadas.
    \item \textbf{Autenticación}: Credenciales de servicio GCP en formato JSON.
    \item \textbf{Versión de Python}: 3.9 o superior (predeterminado en Colab).
    \item \textbf{Sistema operativo}: Ubuntu 20.04+ (base de Colab).
\end{itemize}

\section{Paso a Paso de Configuración y Desarrollo}

\subsection{Paso 1: Configuración Inicial del Entorno Colab}
\begin{enumerate}
    \item Abre Google Colab: \url{https://colab.research.google.com/}
    \item Crea un nuevo notebook: \textit{Archivo → Nuevo notebook}
    \item Cambia el entorno de ejecución: \textit{Runtime → Change runtime type → GPU (T4)}
    \item Verifica la GPU asignada:
\end{enumerate}

\begin{lstlisting}[style=colabstyle, language=Python]
!nvidia-smi -L
\end{lstlisting}

\subsection{Paso 2: Instalación de Dependencias}
\begin{lstlisting}[style=colabstyle, language=Python]
# Actualizar pip e instalar ADK
!pip install --upgrade pip
!pip install google-cloud-aiagent
!pip install google-cloud-aiplatform

# Instalar dependencias adicionales
!pip install langchain
!pip install google-generativeai
!pip install langchain-google-vertexai
\end{lstlisting}

\subsection{Paso 3: Autenticación en Google Cloud}
\begin{lstlisting}[style=colabstyle, language=Python]
from google.colab import auth
auth.authenticate_user()

# Montar Google Drive (opcional)
from google.colab import drive
drive.mount('/content/drive')

# Configurar variables de entorno
import os
os.environ['GOOGLE_CLOUD_PROJECT'] = 'tu-project-id'  # Reemplaza con tu ID

# Autenticar con cuenta de servicio (alternativa)
# Sube tu archivo JSON de credenciales
from google.cloud import aiplatform
from google.oauth2 import service_account

credentials = service_account.Credentials.from_service_account_file(
    '/content/credenciales.json'  # Ruta a tu archivo JSON
)
aiplatform.init(project='tu-project-id', credentials=credentials)
\end{lstlisting}

\subsection{Paso 4: Verificar Versiones y Accesos}
\begin{lstlisting}[style=colabstyle, language=Python]
import sys
print(f"Python: {sys.version}")

import google.cloud.aiagent as aiagent
print(f"ADK Version: {aiagent.__version__}")

# Verificar APIs habilitadas
!gcloud services list --enabled --project=$GOOGLE_CLOUD_PROJECT
\end{lstlisting}

\subsection{Paso 5: Crear un Agente Básico con ADK}
\begin{lstlisting}[style=colabstyle, language=Python]
from google.cloud import aiagent
from google.cloud.aiagent import components

# 1. Inicializar el agente
agent = aiagent.Agent(
    name="mi-primer-agente",
    model="gemini-1.5-pro",  # o "gemini-1.5-flash"
    instructions="Eres un asistente util especializado en analisis de datos."
)

# 2. Agregar herramientas (ejemplo: busqueda web)
search_tool = components.WebSearchTool(
    api_key="TU_API_KEY",  # Reemplazar con clave real
    custom_search_engine_id="TU_CSE_ID"
)
agent.add_tool(search_tool)

# 3. Agregar memoria
memory = components.ConversationMemory(
    max_turns=10,
    storage_path="/content/memory.json"
)
agent.add_component(memory)

# 4. Configurar el planificador
planner = components.ReActPlanner()
agent.add_component(planner)
\end{lstlisting}

\subsection{Paso 6: Probar el Agente}
\begin{lstlisting}[style=colabstyle, language=Python]
# Probar con una consulta simple
response = agent.run("Cual es el precio actual de Bitcoin?")
print("Respuesta del agente:", response.text)

# Ver historial de conversacion
history = memory.get_history()
for turn in history:
    print(f"{turn.role}: {turn.content}")
\end{lstlisting}

\subsection{Paso 7: Despliegue en Vertex AI (Opcional)}
\begin{lstlisting}[style=colabstyle, language=Python]
# Exportar el agente
agent.export_to_vertex(
    project_id=os.environ['GOOGLE_CLOUD_PROJECT'],
    region="us-central1",
    display_name="MiAgenteColab"
)

# O crear endpoint directo
endpoint = agent.deploy(
    machine_type="n1-standard-4",
    min_replica_count=1,
    max_replica_count=3
)
print(f"Endpoint creado: {endpoint.resource_name}")
\end{lstlisting}

\section{Consideraciones Finales}
\begin{itemize}
    \item \textbf{Costos}: Monitorea el uso en Google Cloud Console. Los modelos Gemini y Vertex AI tienen costos asociados.
    \item \textbf{Límites de Colab}: Las sesiones gratuitas expiran a las 12 horas. Guarda el estado en Drive.
    \item \textbf{Mejores prácticas}:
    \begin{enumerate}
        \item Guarda el notebook frecuentemente (Ctrl+S).
        \item Usa celdas de markdown para documentación.
        \item Prueba con consultas simples antes de implementar lógica compleja.
        \item Revoca las credenciales de servicio cuando no las uses.
    \end{enumerate}
    \item \textbf{Solución de problemas}:
    \begin{itemize}
        \item Error de autenticación: Verifica que las APIs estén habilitadas.
        \item Falta de memoria: Reduce el tamaño del modelo o usa batch processing.
        \item Timeout: Usa \texttt{!kill -9 -1} para reiniciar el entorno.
    \end{itemize}
\end{itemize}

\section{Recursos Adicionales}
\begin{itemize}
    \item Documentación oficial ADK: \url{https://cloud.google.com/agent-development-kit}
    \item Repositorio GitHub: \url{https://github.com/google-cloud-ai/agent-development-kit}
    \item Ejemplos de Colab: \url{https://github.com/googlecolab/colabtools}
    \item Foro de soporte: \url{https://stackoverflow.com/questions/tagged/google-colab}
\end{itemize}

\end{document}