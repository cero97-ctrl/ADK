\documentclass[11pt]{article}
\usepackage[utf8]{inputenc}
\usepackage[spanish]{babel}
\usepackage{geometry}
\geometry{a4paper, margin=2cm}
\usepackage{listings}
\usepackage{xcolor}
\usepackage{hyperref}
\usepackage{enumitem}
\usepackage{graphicx}

\definecolor{codebackground}{rgb}{0.95,0.95,0.92}
\definecolor{codecomment}{rgb}{0.5,0.5,0.5}
\definecolor{codekeyword}{rgb}{0,0,0.5}
\definecolor{codestring}{rgb}{0.58,0,0.82}

\lstdefinestyle{localstyle}{
    backgroundcolor=\color{codebackground},
    commentstyle=\color{codecomment},
    keywordstyle=\color{codekeyword},
    numberstyle=\tiny\color{codecomment},
    stringstyle=\color{codestring},
    basicstyle=\ttfamily\footnotesize,
    breakatwhitespace=false,
    breaklines=true,
    captionpos=b,
    keepspaces=true,
    numbers=left,
    numbersep=5pt,
    showspaces=false,
    showstringspaces=false,
    showtabs=false,
    tabsize=2,
    frame=single,
    framerule=0.5pt
}

\title{Guía para Desarrollo Local de Agentes de IA con Google Agent Development Kit}
\author{}
\date{\today}

\begin{document}

\maketitle

\section{Requerimientos del Sistema}
\subsection{Requerimientos de Hardware Mínimos y Recomendados}
\begin{table}[h]
\centering
\begin{tabular}{|l|p{6cm}|p{6cm}|}
\hline
\textbf{Componente} & \textbf{Mínimo} & \textbf{Recomendado} \\
\hline
Procesador (CPU) & Intel i5 de 8ª generación o AMD equivalente & Intel i7/i9 de 11ª+ generación o AMD Ryzen 7/9 \\
\hline
Memoria RAM & 16 GB DDR4 & 32 GB DDR4 o superior \\
\hline
Almacenamiento & 256 GB SSD (50 GB libres) & 512 GB NVMe SSD o superior \\
\hline
GPU (Opcional) & Integrada & NVIDIA RTX 3060+ con 8+ GB VRAM (para modelos locales) \\
\hline
Sistema Operativo & Windows 10, macOS 10.15+, Ubuntu 20.04+ & Ubuntu 22.04 LTS o Windows 11 con WSL2 \\
\hline
Conexión Internet & Banda ancha para descargas & Estable para APIs en la nube \\
\hline
\end{tabular}
\caption{Especificaciones de hardware para desarrollo local}
\end{table}

\subsection{Requerimientos de Software Esenciales}
\begin{itemize}[leftmargin=*]
\item \textbf{Sistema Operativo}: 
  \begin{itemize}
  \item Windows 10/11 con WSL2 (Windows Subsystem for Linux 2)
  \item Ubuntu 20.04 LTS o superior (recomendado)
  \item macOS 12 Monterey o superior
  \end{itemize}
  
\item \textbf{Entorno Python}: 
  \begin{itemize}
  \item Python 3.9, 3.10 o 3.11 (3.12 puede tener incompatibilidades)
  \item pip 23.0 o superior
  \item virtualenv o conda para manejo de entornos
  \end{itemize}

\item \textbf{Herramientas de Desarrollo}:
  \begin{itemize}
  \item Git 2.30+ para control de versiones
  \item Docker Desktop (opcional, para contenedores)
  \item Visual Studio Code o PyCharm IDE
  \end{itemize}

\item \textbf{Google Cloud SDK}:
  \begin{itemize}
  \item CLI de gcloud instalada y configurada
  \item Cuenta de Google Cloud Platform activa
  \item Proyecto GCP creado con facturación habilitada
  \end{itemize}

\item \textbf{Controladores GPU (Opcional)}:
  \begin{itemize}
  \item NVIDIA CUDA Toolkit 11.8 o 12.0
  \item cuDNN compatible con la versión de CUDA
  \end{itemize}
\end{itemize}

\section{Configuración Inicial del Entorno Local}

\subsection{Paso 1: Instalación del Sistema Base}
\begin{enumerate}
\item \textbf{Windows con WSL2 (si aplica)}:
\begin{lstlisting}[style=localstyle, language=bash]
# En PowerShell como Administrador
wsl --install
wsl --set-default-version 2
# Instalar Ubuntu desde Microsoft Store
wsl -l -v  # Verificar instalacion
\end{lstlisting}

\item \textbf{Actualización del sistema (Linux/macOS)}:
\begin{lstlisting}[style=localstyle, language=bash]
# Ubuntu/Debian
sudo apt update && sudo apt upgrade -y
sudo apt install build-essential curl git -y

# macOS
xcode-select --install
brew update && brew upgrade
\end{lstlisting}
\end{enumerate}

\subsection{Paso 2: Instalación y Configuración de Python}
\begin{lstlisting}[style=localstyle, language=bash]
# Verificar Python existente
python3 --version
pip3 --version

# Instalar Python 3.11 si es necesario (Ubuntu)
sudo apt install python3.11 python3.11-venv python3.11-dev -y

# Crear entorno virtual
python3.11 -m venv ~/adk-env
source ~/adk-env/bin/activate  # Linux/macOS
# En Windows (PowerShell): ~\adk-env\Scripts\Activate.ps1

# Actualizar pip y herramientas basicas
pip install --upgrade pip setuptools wheel
\end{lstlisting}

\subsection{Paso 3: Instalación de Google Cloud SDK}
\begin{lstlisting}[style=localstyle, language=bash]
# Linux (Ubuntu)
echo "deb [signed-by=/usr/share/keyrings/cloud.google.gpg] https://packages.cloud.google.com/apt cloud-sdk main" | sudo tee -a /etc/apt/sources.list.d/google-cloud-sdk.list
curl https://packages.cloud.google.com/apt/doc/apt-key.gpg | sudo apt-key --keyring /usr/share/keyrings/cloud.google.gpg add -
sudo apt update && sudo apt install google-cloud-cli -y

# macOS
brew install --cask google-cloud-sdk

# Windows (PowerShell)
(New-Object Net.WebClient).DownloadFile("https://dl.google.com/dl/cloudsdk/channels/rapid/GoogleCloudSDKInstaller.exe", "$env:Temp\GoogleCloudSDKInstaller.exe")
& "$env:Temp\GoogleCloudSDKInstaller.exe"

# Inicializar y autenticar
gcloud init
gcloud auth application-default login
gcloud auth login
\end{lstlisting}

\subsection{Paso 4: Configuración del Proyecto GCP}
\begin{lstlisting}[style=localstyle, language=bash]
# Crear nuevo proyecto o usar existente
gcloud projects create mi-proyecto-adk --name="Proyecto ADK Local"
gcloud config set project mi-proyecto-adk

# Habilitar APIs necesarias
gcloud services enable aiplatform.googleapis.com
gcloud services enable cloudresourcemanager.googleapis.com
gcloud services enable iamcredentials.googleapis.com

# Configurar region por defecto
gcloud config set compute/region us-central1
gcloud config set compute/zone us-central1-a

# Crear cuenta de servicio
gcloud iam service-accounts create adk-local-sa \
    --display-name="Cuenta de servicio ADK Local"

# Otorgar permisos
gcloud projects add-iam-policy-binding mi-proyecto-adk \
    --member="serviceAccount:adk-local-sa@mi-proyecto-adk.iam.gserviceaccount.com" \
    --role="roles/aiplatform.user"

# Generar clave de servicio (JSON)
gcloud iam service-accounts keys create ~/adk-key.json \
    --iam-account=adk-local-sa@mi-proyecto-adk.iam.gserviceaccount.com

# Establecer variable de entorno
export GOOGLE_APPLICATION_CREDENTIALS=~/adk-key.json
# En Windows: $env:GOOGLE_APPLICATION_CREDENTIALS="C:\Users\Usuario\adk-key.json"
\end{lstlisting}

\section{Instalación y Configuración del ADK}

\subsection{Paso 5: Instalación del Agent Development Kit}
\begin{lstlisting}[style=localstyle, language=bash]
# Con entorno virtual activado
pip install google-cloud-aiagent
pip install google-cloud-aiplatform>=1.38

# Instalar dependencias adicionales recomendadas
pip install langchain
pip install langchain-google-vertexai
pip install google-generativeai
pip install pandas numpy matplotlib  # Para procesamiento de datos
pip install jupyter notebook  # Para experimentacion interactiva

# Verificar instalacion
python -c "import google.cloud.aiagent as aiagent; print(f'ADK Version: {aiagent.__version__}')"
\end{lstlisting}

\subsection{Paso 6: Configuración del Entorno de Desarrollo}
\begin{lstlisting}[style=localstyle, language=python]
# Crear archivo de configuracion local: config_adk.py
import os
from google.oauth2 import service_account

# Configuracion base
PROJECT_ID = "mi-proyecto-adk"
LOCATION = "us-central1"
MODEL_NAME = "gemini-1.5-pro"  # Alternativa: "gemini-1.5-flash"

# Ruta a las credenciales (ajustar segun tu sistema)
CREDENTIALS_PATH = os.path.expanduser("~/adk-key.json")

# Validar que el archivo existe
if not os.path.exists(CREDENTIALS_PATH):
    raise FileNotFoundError(f"Credenciales no encontradas en: {CREDENTIALS_PATH}")

# Cargar credenciales
credentials = service_account.Credentials.from_service_account_file(
    CREDENTIALS_PATH,
    scopes=["https://www.googleapis.com/auth/cloud-platform"]
)

# Configurar variables de entorno
os.environ["GOOGLE_CLOUD_PROJECT"] = PROJECT_ID
os.environ["GOOGLE_APPLICATION_CREDENTIALS"] = CREDENTIALS_PATH

print("Configuracion cargada correctamente")
print(f"Proyecto: {PROJECT_ID}")
print(f"Modelo: {MODEL_NAME}")
\end{lstlisting}

\section{Desarrollo del Agente de IA}

\subsection{Paso 7: Crear un Agente Básico Local}
\begin{lstlisting}[style=localstyle, language=python]
# archivo: mi_agente_local.py
import os
import sys
from google.cloud import aiplatform
from google.cloud import aiagent
from google.cloud.aiagent import components

# Inicializar Vertex AI
aiplatform.init(
    project=os.environ["GOOGLE_CLOUD_PROJECT"],
    location="us-central1",
    credentials=credentials  # Del paso anterior
)

class MiAgenteLocal:
    def __init__(self, model_name="gemini-1.5-pro"):
        """Inicializar agente con componentes basicos"""
        
        # 1. Crear instancia del agente
        self.agent = aiagent.Agent(
            name="agente-local-demo",
            model=model_name,
            instructions="""Eres un asistente especializado ejecutandose localmente.
            Proporciona respuestas precisas y utiles.
            Si no sabes algo, admitelo honestamente."""
        )
        
        # 2. Agregar memoria de conversacion
        self.memory = components.ConversationMemory(
            max_turns=20,
            storage_path="./conversation_memory.json"
        )
        self.agent.add_component(self.memory)
        
        # 3. Configurar planificador
        self.planner = components.ReActPlanner()
        self.agent.add_component(self.planner)
        
        print(f"Agente inicializado con modelo: {model_name}")
    
    def agregar_herramienta_busqueda(self, api_key=None, cse_id=None):
        """Agregar herramienta de busqueda web (opcional)"""
        if api_key and cse_id:
            search_tool = components.WebSearchTool(
                api_key=api_key,
                custom_search_engine_id=cse_id
            )
            self.agent.add_tool(search_tool)
            print("Herramienta de busqueda agregada")
    
    def agregar_herramienta_codigo(self):
        """Agregar herramienta para ejecutar codigo Python"""
        code_tool = components.CodeInterpreterTool(
            safe_execution=True,
            timeout_seconds=30
        )
        self.agent.add_tool(code_tool)
        print("Herramienta de codigo agregada")
    
    def consultar(self, pregunta):
        """Ejecutar consulta con el agente"""
        print(f"\n[Usuario]: {pregunta}")
        response = self.agent.run(pregunta)
        print(f"\n[Agente]: {response.text}")
        return response
    
    def guardar_estado(self, ruta="./estado_agente.json"):
        """Guardar estado del agente en disco"""
        self.agent.save(ruta)
        print(f"Estado guardado en: {ruta}")
    
    def cargar_estado(self, ruta="./estado_agente.json"):
        """Cargar estado del agente desde disco"""
        if os.path.exists(ruta):
            self.agent.load(ruta)
            print(f"Estado cargado desde: {ruta}")

# Uso basico
if __name__ == "__main__":
    # Crear instancia del agente
    agente = MiAgenteLocal(model_name="gemini-1.5-flash")
    
    # Ejemplo de consulta
    respuesta = agente.consultar(
        "Explica los beneficios de desarrollar agentes de IA localmente"
    )
    
    # Guardar historial
    agente.guardar_estado()
\end{lstlisting}

\subsection{Paso 8: Agente con Funciones Personalizadas}
\begin{lstlisting}[style=localstyle, language=python]
# archivo: agente_personalizado.py
from typing import Dict, Any
import json
from datetime import datetime

class AgentePersonalizado:
    def __init__(self):
        # Herramientas personalizadas
        self.custom_tools = {
            "calculadora": self._herramienta_calculadora,
            "formateador_json": self._herramienta_json,
            "obtener_fecha": self._herramienta_fecha
        }
        
    def _herramienta_calculadora(self, expresion: str) -> str:
        """Evaluar expresion matematica segura"""
        try:
            # Solo operaciones matematicas basicas
            allowed_chars = set("0123456789+-*/(). ")
            if all(c in allowed_chars for c in expresion):
                result = eval(expresion)
                return f"Resultado: {result}"
            else:
                return "Error: Expresion contiene caracteres no permitidos"
        except Exception as e:
            return f"Error en calculo: {str(e)}"
    
    def _herramienta_json(self, data: str) -> str:
        """Formatear y validar JSON"""
        try:
            parsed = json.loads(data)
            return json.dumps(parsed, indent=2, ensure_ascii=False)
        except json.JSONDecodeError as e:
            return f"JSON invalido: {str(e)}"
    
    def _herramienta_fecha(self, formato: str = "%Y-%m-%d") -> str:
        """Obtener fecha actual"""
        return datetime.now().strftime(formato)

# Integracion con ADK
def crear_agente_completo():
    from google.cloud import aiagent
    
    agent = aiagent.Agent(
        name="agente-avanzado",
        model="gemini-1.5-pro",
        instructions="Usa las herramientas disponibles para ayudar al usuario"
    )
    
    # Agregar herramientas personalizadas via funciones
    personalizado = AgentePersonalizado()
    
    # Convertir herramientas personalizadas al formato ADK
    for name, func in personalizado.custom_tools.items():
        tool_wrapper = aiagent.Tool.from_function(
            func=func,
            name=name,
            description=f"Herramienta para {name}"
        )
        agent.add_tool(tool_wrapper)
    
    return agent
\end{lstlisting}

\section{Pruebas y Validación}

\subsection{Paso 9: Suite de Pruebas Local}
\begin{lstlisting}[style=localstyle, language=python]
# archivo: test_agente.py
import unittest
from mi_agente_local import MiAgenteLocal

class TestAgenteLocal(unittest.TestCase):
    
    @classmethod
    def setUpClass(cls):
        """Inicializar agente una vez para todas las pruebas"""
        cls.agente = MiAgenteLocal(model_name="gemini-1.5-flash")
    
    def test_inicializacion(self):
        """Verificar que el agente se inicializa correctamente"""
        self.assertIsNotNone(self.agente.agent)
        self.assertIsNotNone(self.agente.memory)
    
    def test_consulta_basica(self):
        """Probar consulta simple"""
        respuesta = self.agente.consultar("Hola, como estas?")
        self.assertIsNotNone(respuesta)
        self.assertIsInstance(respuesta.text, str)
        self.assertGreater(len(respuesta.text), 10)
    
    def test_memoria_conversacion(self):
        """Verificar que la memoria funciona"""
        pregunta1 = "Mi nombre es Carlos"
        self.agente.consultar(pregunta1)
        
        pregunta2 = "Cual es mi nombre?"
        respuesta = self.agente.consultar(pregunta2)
        
        # Verificar que recuerda el contexto
        self.assertIn("Carlos", respuesta.text)
    
    def test_guardado_carga(self):
        """Probar persistencia del agente"""
        ruta_test = "./test_estado.json"
        
        # Guardar estado
        self.agente.guardar_estado(ruta_test)
        
        # Crear nuevo agente y cargar estado
        nuevo_agente = MiAgenteLocal()
        nuevo_agente.cargar_estado(ruta_test)
        
        self.assertIsNotNone(nuevo_agente.agent)
        
        # Limpiar archivo de prueba
        import os
        if os.path.exists(ruta_test):
            os.remove(ruta_test)

if __name__ == "__main__":
    unittest.main(verbosity=2)
\end{lstlisting}

\subsection{Paso 10: Ejecución y Monitoreo}
\begin{lstlisting}[style=localstyle, language=bash]
# Ejecutar el agente en modo interactivo
python -i mi_agente_local.py

# Ejecutar pruebas unitarias
python -m pytest test_agente.py -v

# Monitorear uso de recursos durante ejecucion
# Linux:
htop  # En otra terminal
nvidia-smi -l 1  # Para GPU NVIDIA

# Windows:
# Abrir Administrador de Tareas -> Rendimiento

# Ejecutar con logging detallado
import logging
logging.basicConfig(level=logging.DEBUG)
\end{lstlisting}

\section{Despliegue y Producción}

\subsection{Paso 11: Configuración para Producción}
\begin{lstlisting}[style=localstyle, language=python]
# archivo: deployment_config.py
import yaml

config_produccion = {
    "version": "1.0",
    "agent": {
        "name": "agente-produccion",
        "model": "gemini-1.5-pro",
        "max_concurrent_requests": 10,
        "timeout_seconds": 60,
        "rate_limit": {
            "requests_per_minute": 100,
            "requests_per_day": 10000
        }
    },
    "monitoring": {
        "enable_logging": True,
        "log_level": "INFO",
        "metrics": ["latency", "success_rate", "token_usage"],
        "alerting": {
            "email": "admin@example.com",
            "thresholds": {
                "error_rate": 0.05,
                "p99_latency": 5000  # ms
            }
        }
    },
    "security": {
        "api_key_required": True,
        "cors_origins": ["https://tudominio.com"],
        "input_validation": True,
        "output_sanitization": True
    }
}

# Guardar configuracion
with open("config_produccion.yaml", "w") as f:
    yaml.dump(config_produccion, f, default_flow_style=False)
\end{lstlisting}

\subsection{Paso 12: Contenerización (Opcional)}
\begin{lstlisting}[style=localstyle, language=bash]
# Dockerfile
FROM python:3.11-slim

WORKDIR /app

# Instalar dependencias del sistema
RUN apt-get update && apt-get install -y \
    curl \
    git \
    && rm -rf /var/lib/apt/lists/*

# Copiar requirements
COPY requirements.txt .
RUN pip install --no-cache-dir -r requirements.txt

# Copiar codigo de la aplicacion
COPY . .

# Crear usuario no root
RUN useradd -m -u 1000 appuser && chown -R appuser:appuser /app
USER appuser

# Variables de entorno
ENV GOOGLE_APPLICATION_CREDENTIALS=/app/credentials/key.json
ENV PYTHONPATH=/app
ENV PORT=8080

# Exponer puerto
EXPOSE 8080

# Comando de inicio
CMD ["python", "app/main.py"]
\end{lstlisting}

\section{Solución de Problemas Comunes}

\begin{table}[h]
\centering
\begin{tabular}{|p{5cm}|p{5cm}|p{5cm}|}
\hline
\textbf{Problema} & \textbf{Causa Probable} & \textbf{Solución} \\
\hline
Error de autenticación & Credenciales no configuradas o inválidas & Verificar GOOGLE\_APPLICATION\_CREDENTIALS, regenerar clave JSON \\
\hline
Falta de memoria & Modelo muy grande o muchas herramientas & Reducir tamaño de modelo, usar gemini-1.5-flash, aumentar swap \\
\hline
Lentitud en respuestas & Conexión lenta a APIs de Google & Usar modelo local (si GPU disponible), optimizar consultas \\
\hline
ImportError: módulo no encontrado & Entorno virtual no activado o dependencias faltantes & Activar entorno virtual, pip install -r requirements.txt \\
\hline
Límites de cuota excedidos & Uso excesivo de APIs de Google & Solicitar aumento de cuota, implementar caché, monitorear uso \\
\hline
Problemas con WSL2 en Windows & Configuración incorrecta de WSL2 & wsl --update, aumentar memoria asignada en .wslconfig \\
\hline
\end{tabular}
\caption{Problemas comunes y soluciones en desarrollo local}
\end{table}

\section{Recursos y Referencias}

\begin{itemize}
\item \textbf{Documentación Oficial}:
  \begin{itemize}
  \item Google ADK: \url{https://cloud.google.com/agent-development-kit}
  \item Vertex AI: \url{https://cloud.google.com/vertex-ai}
  \item Gemini API: \url{https://cloud.google.com/vertex-ai/generative-ai}
  \end{itemize}
  
\item \textbf{Repositorios de Ejemplo}:
  \begin{itemize}
  \item GitHub ADK: \url{https://github.com/google-cloud-ai/agent-development-kit}
  \item Ejemplos oficiales: \url{https://github.com/GoogleCloudPlatform/ai-agent-examples}
  \end{itemize}
  
\item \textbf{Comunidad y Soporte}:
  \begin{itemize}
  \item Stack Overflow: \url{https://stackoverflow.com/questions/tagged/google-cloud-ai}
  \item Google Cloud Community: \url{https://www.googlecloudcommunity.com}
  \end{itemize}

\item \textbf{Herramientas Adicionales}:
  \begin{itemize}
  \item LangChain: \url{https://python.langchain.com}
  \item LlamaIndex: \url{https://www.llamaindex.ai}
  \item Ollama (para modelos locales): \url{https://ollama.ai}
  \end{itemize}
\end{itemize}

\section{Conclusión}

Desarrollar agentes de IA con Google ADK en una PC local ofrece mayor control, flexibilidad y privacidad. Esta guía proporciona los pasos necesarios desde la configuración del hardware hasta el despliegue de agentes funcionales. Para mantener el entorno actualizado, se recomienda:

\begin{enumerate}
\item Actualizar regularmente los paquetes Python: \texttt{pip list --outdated}
\item Revisar los cambios en las APIs de Google Cloud
\item Mantener backups de las configuraciones y credenciales
\item Monitorear el uso y costos en Google Cloud Console
\end{enumerate}

Con esta configuración, podrás desarrollar, probar y refinar agentes de IA complejos directamente desde tu computadora personal antes de desplegarlos en entornos de producción.

\end{document}